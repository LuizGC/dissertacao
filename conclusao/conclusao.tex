\section{Conclusão}

As redes veiculares tem um potencial muito amplo para realização de pesquisas. Nas pesquisas é de fundamental importancia conhecer o tráfego de dados em aplicações de segurança no trânsito. As redes veiculares em sua configuração híbridas até o momento não foram totalmente exploradas e podem ser aprimoradas.

Os agentes móveis adicionam as redes veiculares maior flexibilidade para trafegar dados. Através do agente os dados podem ser transportados levando em consideração o contexto, assim o agente decide se deve ou não migrar para outro nó. Outro ponto importante é que o agente pode decidir dependendo do contexto em usar outras formas de comunicação. 

Nas simulações é possivel observar quea infraestrutura fixa melhora a performance do agente em comparação a uma rede sem a infraestrutura. Porém o impacto da infraestrutura em redes densas é menor. Essa informação é importante para auxiliar na escolha das regiões onde serão colocadas as infraestrturas. Regiões mais densas de veículos podem receber menos infraestruturas e regiões menos densas recebem mais infraestrutura. 

No experimento com o protótipo, o ZigBee se mostrou viável para transportar agente de software simples. Porém os agentes possuem uma complexidade maior que as informações transportadas no trabalho \cite{santanaMestrado:2014} que obteve latência média de 64 ms, nessa trabalho a latência média foi de 124ms com infraestrutura e 126 sem infraestrutura. Assim a latência se mostrou alta para aplicações anti-colisão entre veículos.

A taxa de perda do agente com infraestrutura foi de 10,23\% sendo menor que a taxa de 21,43\% obtida pela experimento sem infraestrutura. Essa informação mostra que o uso de infraestrutura como suporte na migração ajuda a diminuir a perda do agente.
